
\documentclass[a4paper,12pt]{article}
\author{}
\date{}
\usepackage[papersize={216mm,330mm},tmargin=20mm,bmargin=20mm,lmargin=20mm,rmargin=20mm]{geometry}
\usepackage[brazil]{babel}
\usepackage[utf8]{inputenc}
\usepackage{amsmath,amssymb} %,mathabx}%\for eqref
\usepackage{lscape}
\usepackage{graphicx}
\usepackage[colorinlistoftodos]{todonotes}
\usepackage{fancyhdr}
\usepackage{tasks}
\usepackage{float}
\usepackage{multicol}
\usepackage{color}
\usepackage{ragged2e} % justifying
\usepackage{colortbl}
\usepackage{cancel}
\usepackage{stackengine}
\usepackage{mathtools}
\usepackage{tikz}   %create images
\usepackage{tkz-fct}    %create graphics

%====================================================
%============= INFORMACAO DE ENTRADA=================
%====================================================

\def\nomedoaluno{Lucas Manoel Arruda Pinheiro}
\def\matricula{20230046509}

%IMD1002 - Análise Combinatória
%IMD0024 - Cálculo 1
%IMD0034 - Vetores e Geometria Analítica
\def\coddisciplina{IMD0034}
\def\nomedisciplina{VETORES E GEOMETRIA ANALÍTICA}
\def\codturma{01}
\def\codatividade{U1\_VGATRAB20}

%====================================================
%====================== FIM =========================
%====================================================

\pagestyle{fancy}
\fancyhf{}
\lhead{Aluno: \nomedoaluno\\ Matrícula: \matricula}
\chead{\thepage}
\rhead{Instituto Metrópole Digital - UFRN \\
Turma \codturma}
\lfoot{\nomedisciplina}
\cfoot{Atividade \codatividade}
\rfoot{Prof. Samyr Jácome}

\title{
\vspace{-2cm}
\Large \textbf{Universidade Federal do Rio Grande do Norte}\\ 
Instituto Metrópole Digital \\ 
\coddisciplina $\;$ - \nomedisciplina \\ 
TURMA \codturma \\
\vspace{5mm} \Large\textbf{ATIVIDADE \codatividade} \\
\normalsize Natal-RN, \today\\
\vspace{0.7cm} \large \textit{Prof. Samyr Jácome}\\

\justifying
\vspace{0.5cm} \hspace{-0.82cm}
\begin{minipage}{.6\linewidth}
    \large \textbf{Aluno:} \nomedoaluno
\end{minipage}
\begin{minipage}{.4\linewidth}
    \begin{flushright}
        \large \textbf{Matrícula:} \matricula
    \end{flushright}
\end{minipage}
}

% Tirar a identação do paragrafo no texto
\def\tirarident{\setlength{\parindent}{0cm}} % padrão 15pt.

\setlength{\marginparwidth}{2cm}

%---------------------------------------------------------------
%---------------COMANDO PARA SETA DE ESCALONAMENTO--------------
%---------------------------------------------------------------

\newcommand{\seta}[3][-0.05cm]{%
  \stackon[#1]{
    $\xrightarrow{\mathmakebox[\setawidth]{}}$}{
    \scriptsize{$#2 \,\leftarrow\, #3$}
    }
}

\newcommand{\troca}[3][-0.05cm]{%
  \stackon[#1]{
    $\xrightarrow{\mathmakebox[\setawidth]{}}$}{
    \scriptsize{$#2 \,\leftrightarrow\, #3$}
    }
}

\newlength{\setawidth}% row operation width
\AtBeginDocument{\setlength{\setawidth}{2.0cm}}

%---------------------------------------------------------------
%--------------------------FIM----------------------------------
%---------------------------------------------------------------

\begin{document}
\maketitle

\vspace{-2cm}
\section*{Exercícios}

\tirarident

\textbf{Questão 07, pág. 23:}
\begin{alignat*}{2}
    \begin{bmatrix}
        1 & -1 & 2 & -1 & -1 \\
        2 & 1 & -2 & -2 & -2 \\
        -1 & 2 & -4 & 1 & 1 \\
        3 & 0 & 0 & -3 & -3 \\
    \end{bmatrix}
    &\!\begin{aligned}
        &\seta{L_2}{L_2 - 2L_1}\\
        &\seta{L_3}{L_3 + L_1}\\
        &\seta{L_4}{L_4 - 3L_1}
    \end{aligned}
    \begin{bmatrix}
        1 & -1 & 2 & -1 & -1 \\
        0 & 3 & -6 & 0 & 0 \\
        0 & 1 & -2 & 0 & 0 \\
        0 & 3 & -6 & 0 & 0 \\
    \end{bmatrix}
    \\
    &\!\begin{aligned}
        &\seta{L_2}{L_2/3}\\
        &\seta{L_4}{L_4/3}
    \end{aligned}
    \begin{bmatrix}
        1 & -1 & 2 & -1 & -1 \\
        0 & 1 & -2 & 0 & 0 \\
        0 & 1 & -2 & 0 & 0 \\
        0 & 1 & -1 & 0 & 0 \\
    \end{bmatrix}
    \\
    &\!\begin{aligned}
        &\seta{L_3}{L_3 - L_2}\\
        &\seta{L_4}{L_4 - L_2}
    \end{aligned}
    \begin{bmatrix}
        1 & -1 & 2 & -1 & -1 \\
        0 & 1 & -2 & 0 & 0 \\
        0 & 0 & 0 & 0 & 0 \\
        0 & 0 & 0 & 0 & 0 \\
    \end{bmatrix}
    \\
    &\!\begin{aligned}
        &\seta{L_1}{L_1 + L_2}
    \end{aligned}
    \begin{bmatrix}
        1 & 0 & 0 & -1 & -1 \\
        0 & 1 & -2 & 0 & 0 \\
        0 & 0 & 0 & 0 & 0 \\
        0 & 0 & 0 & 0 & 0 \\
    \end{bmatrix}
\end{alignat*}

Não podemos mais continuar com o escalonamento e, portanto, a resposta é: (x,y,z,w) = (r, 2s, s, r+1)    

\textbf{Questão 09, pág. 23:}

\begin{alignat*}{2}
    \begin{bmatrix}
        1 & 1 & 2 & 8\\
        -1 & -2 & 3 & 1\\
        3 & -7 & 4 & 10
    \end{bmatrix}
    &\!\begin{aligned}
        &\seta{L_2}{L_2 + L_1}\\
        &\seta{L_3}{L_3 - 3L_1}
    \end{aligned}
    \begin{bmatrix}
        1 & 1 & 2 & 8\\
        0 & -1 & 5 & 9\\
        0 & -10 & -2 & -14
    \end{bmatrix}
    \\
    &\!\begin{aligned}
        &\seta{L_3}{L_3 - 10L_2}
    \end{aligned}
    \begin{bmatrix}
        1 & 1 & 2 & 8\\
        0 & -1 & 5 & 9\\
        0 & 0 & 26 & 52
    \end{bmatrix}
    &\!\begin{aligned}
        &\seta{L_3}{L_3 - 10L_2}
    \end{aligned}
    \begin{bmatrix}
        1 & 1 & 2 & 8\\
        0 & -1 & 5 & 9\\
        0 & 0 & 1 & 2
    \end{bmatrix}
\end{alignat*}

Reescrevemos essa matriz como sistema linear:

\begin{equation}
    \begin{cases}
        x_1 +  x_2 +  2x_3 = 8 \\
        -x_2  + 5x_ 3 = 9 \\
        x_3 = 2
    \end{cases}
\end{equation}

Substituimos $x_3 = 2$ na segunda equação do sistema:
\begin{align*}
    -x_2  + 5x_ 3 = 9 &\implies -x_2  + 5\cdot 2 = 9\\
    &\implies -x_2  + 10 = 9\\
    &\implies x_ 2 = 1
\end{align*}

Substituimos $x_2 = 1$ e $x_3 = 2$ na primeira equação do sistema:
\begin{align*}
    x_1 +  x_2 +  2x_3 = 8 &\implies x_1 + 1 + 2\cdot2 = 8\\
    &\implies x_1 + 1 + 4 = 8\\
    &\implies x_1 = 3
\end{align*}

Assim, podemos afirmar que o conjunto solução do sistema apresentado é $\{3, 1, 2\}$ \\

\textbf{Questão 19, pág. 23:}

A matriz aumentada desse sistema é:

\begin{alignat*}{2}
    \begin{bmatrix}
        3 & 1 & 1 & 1 & 0\\
        5 & -1 & 1 & -1 & 0
    \end{bmatrix}
\end{alignat*}

Resolvendo por eliminação de Gauss-Jordan:

\begin{alignat*}{2}
    \begin{bmatrix}
        3 & 1 & 1 & 1 & 0\\
        5 & -1 & 1 & -1 & 0
    \end{bmatrix}
    &\!\begin{aligned}
        &\seta{L_1}{L_1 \cdot 5}\\
        &\seta{L_2}{L_2 \cdot 3}
    \end{aligned}
    \begin{bmatrix}
        15 & 5 & 5 & 5 & 0\\
        15 & -3 & 3 & -3 & 0
    \end{bmatrix}
    \\
    &\!\begin{aligned}
        &\seta{L_2}{L_2 - L_1}
    \end{aligned}
    \begin{bmatrix}
        15 & 5 & 5 & 5 & 0\\
        0 & -8 & -2 & -8 & 0
    \end{bmatrix}
    \\
    &\!\begin{aligned}
        &\seta{L_1}{4L_1 / 5}\\
        &\seta{L_2}{L_2 / (-2)}
    \end{aligned}
    \begin{bmatrix}
        12 & 4 & 4 & 4 & 0\\
        0 & 4 & 1 & 4 & 0
    \end{bmatrix}
    \\
    &\!\begin{aligned}
        &\seta{L_1}{L_1 - L_2}\\
    \end{aligned}
    \begin{bmatrix}
        12 & 0 & 3 & 0 & 0\\
        0 & 4 & 1 & 4 & 0
    \end{bmatrix}
    \\
    &\!\begin{aligned}
        &\seta{L_1}{L_1/3}\\
    \end{aligned}
    \begin{bmatrix}
        4 & 0 & 1 & 0 & 0\\
        0 & 4 & 1 & 4 & 0
    \end{bmatrix}
\end{alignat*}

Não podemos mais continuar com o escalonamento e, portanto, a resposta é: ($x_1, x_2, x_3, x_ 4$) = ($-\frac{r}{4}, -\frac{r+4s}{4},r, s\ $)

\textbf{Questão 32, pág. 24:}

\begin{alignat*}{2}
    \begin{bmatrix}
        2 & 1 & 3\\
        0 & -2 & -29\\
        3 & 4 & 5
    \end{bmatrix}
    &\!\begin{aligned}
        &\seta{L_3}{L_3 \cdot 2}
    \end{aligned}
    \begin{bmatrix}
        2 & 1 & 3\\
        0 & -2 & -29\\
        6 & 8 & 10
    \end{bmatrix}
    \\
    &\!\begin{aligned}
        &\seta{L_3}{L_3 - 3L_1}
    \end{aligned}
    \begin{bmatrix}
        2 & 1 & 3\\
        0 & -2 & -29\\
        0 & 5 & 1
    \end{bmatrix}
    \\
    &\!\begin{aligned}
        &\seta{L_3}{L_3 \cdot 2}
    \end{aligned}
    \begin{bmatrix}
        6 & 3 & 9\\
        0 & -2 & -29\\
        0 & 10 & 2
    \end{bmatrix}
    \\
    &\!\begin{aligned}
        &\seta{L_3}{L_3 + 5L_2}
    \end{aligned}
    \begin{bmatrix}
        6 & 3 & 9\\
        0 & -2 & -29\\
        0 & 0 & -143
    \end{bmatrix}    
    \\
    &\!\begin{aligned}
        &\seta{L_3}{L_3/-143}
    \end{aligned}
    \begin{bmatrix}
        6 & 3 & 9\\
        0 & -2 & -29\\
        0 & 0 & 1
    \end{bmatrix}    
    \\
    &\!\begin{aligned}
        &\seta{L_2}{L_2 + 29L3}\\
        &\seta{L_1}{L_1 - 9L3}
    \end{aligned}
    \begin{bmatrix}
        6 & 3 & 0 \\
        0 & -2 & 0\\
        0 & 0 & 1
    \end{bmatrix} 
\end{alignat*}

\end{document}