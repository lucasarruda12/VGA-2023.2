
\documentclass[a4paper,12pt]{article}
\author{}
\date{}
\usepackage[papersize={216mm,330mm},tmargin=20mm,bmargin=20mm,lmargin=20mm,rmargin=20mm]{geometry}
\usepackage[brazil]{babel}
\usepackage[utf8]{inputenc}
\usepackage{amsmath,amssymb} %,mathabx}%\for eqref
\usepackage{lscape}
\usepackage{graphicx}
\usepackage[colorinlistoftodos]{todonotes}
\usepackage{fancyhdr}
\usepackage{tasks}
\usepackage{float}
\usepackage{multicol}
\usepackage{color}
\usepackage{ragged2e} % justifying
\usepackage{colortbl}
\usepackage{cancel}
\usepackage{stackengine}
\usepackage{mathtools}
\usepackage{tikz}   %create images
\usepackage{tkz-fct}    %create graphics

%====================================================
%============= INFORMACAO DE ENTRADA=================
%====================================================

\def\nomedoaluno{Lucas Manoel Arruda Pinheiro}
\def\matricula{20230046509}

%IMD1002 - Análise Combinatória
%IMD0024 - Cálculo 1
%IMD0034 - Vetores e Geometria Analítica
\def\coddisciplina{IMD0034}
\def\nomedisciplina{VETORES E GEOMETRIA ANALÍTICA}
\def\codturma{01}
\def\codatividade{U1\_VGATRAB40}

%====================================================
%====================== FIM =========================
%====================================================

\pagestyle{fancy}
\fancyhf{}
\lhead{Aluno: \nomedoaluno\\ Matrícula: \matricula}
\chead{\thepage}
\rhead{Instituto Metrópole Digital - UFRN \\
Turma \codturma}
\lfoot{\nomedisciplina}
\cfoot{Atividade \codatividade}
\rfoot{Prof. Samyr Jácome}

\title{
\vspace{-2cm}
\Large \textbf{Universidade Federal do Rio Grande do Norte}\\ 
Instituto Metrópole Digital \\ 
\coddisciplina $\;$ - \nomedisciplina \\ 
TURMA \codturma \\
\vspace{5mm} \Large\textbf{ATIVIDADE \codatividade} \\
\normalsize Natal-RN, \today\\
\vspace{0.7cm} \large \textit{Prof. Samyr Jácome}\\

\justifying
\vspace{0.5cm} \hspace{-0.82cm}
\begin{minipage}{.6\linewidth}
    \large \textbf{Aluno:} \nomedoaluno
\end{minipage}
\begin{minipage}{.4\linewidth}
    \begin{flushright}
        \large \textbf{Matrícula:} \matricula
    \end{flushright}
\end{minipage}
}

% Tirar a identação do paragrafo no texto
\def\tirarident{\setlength{\parindent}{0cm}} % padrão 15pt.

\setlength{\marginparwidth}{2cm}

%---------------------------------------------------------------
%---------------COMANDO PARA SETA DE ESCALONAMENTO--------------
%---------------------------------------------------------------

\newcommand{\seta}[3][-0.05cm]{%
  \stackon[#1]{
    $\xrightarrow{\mathmakebox[\setawidth]{}}$}{
    \scriptsize{$#2 \,\leftarrow\, #3$}
    }
}

\newcommand{\troca}[3][-0.05cm]{%
  \stackon[#1]{
    $\xrightarrow{\mathmakebox[\setawidth]{}}$}{
    \scriptsize{$#2 \,\leftrightarrow\, #3$}
    }
}

\newlength{\setawidth}% row operation width
\AtBeginDocument{\setlength{\setawidth}{2.0cm}}

%---------------------------------------------------------------
%--------------------------FIM----------------------------------
%---------------------------------------------------------------

\begin{document}
\maketitle

\vspace{-2cm}
\section*{Exercícios}

\tirarident

\textbf{Questão 3, pág. 117:}\\

(a)

\begin{alignat*}{2}
  \begin{vmatrix}
      -1 &  5 &  2\\
      0 & 2  & -1\\
      -3 &  1 &  1
  \end{vmatrix}
  &= 0 \cdot 
  \begin{vmatrix}
    5 & 2\\
    1 & 1
  \end{vmatrix} +
  2 \cdot 
  \begin{vmatrix}
    -1 & 2\\
    -3 & 1
  \end{vmatrix} +
  (-1) \cdot 
  \begin{vmatrix}
    -1 & 5\\
    -3 & 1
  \end{vmatrix}
  \\
  &= (-1)^{3} \cdot  0 \cdot (5 - 2) + (-1)^{4} \cdot 2 \cdot ((-1) - (-6)) + (-1)^{5} \cdot (-1) \cdot ((-1) - (-15))\\
  &= 0 + 10 + 14\\
  &= 24
\end{alignat*}

(b)

\begin{alignat*}{2}
  \begin{vmatrix}
      -1 &  5 &  2\\
      0 & 2  & -1\\
      -3 &  1 &  1
  \end{vmatrix}
  &=
  \begin{vmatrix}
      -1 &  5 &  2\\
      0 & 2  & -1\\
      0 &  -14 &  -5
  \end{vmatrix}
  \\
  &=
  \begin{vmatrix}
      -1 &  5 &  2\\
      0 & 2  & -1\\
      0 &  0 &  -12
  \end{vmatrix}
\end{alignat*}

Logo, o determinante dessa matriz é $(-1) \cdot 2 \cdot (-12) = 24$.\\

\textbf{Questão 3, pág. 117:}\\

(a)

\begin{alignat*}{2}
  \begin{vmatrix}
      3 &  0 &  -1\\
      1 & 1  & 1\\
      0 &  4 &  2
  \end{vmatrix}
  &= 3 \cdot
  \begin{vmatrix}
    1  & 1\\
    4 &  2
  \end{vmatrix} +
  0 \cdot
  \begin{vmatrix}
    1  & 1\\
    0 &  2
  \end{vmatrix} +
  (-1) \cdot
  \begin{vmatrix}
    1  & 1\\
    0 &  4
  \end{vmatrix}
  \\
  &= (-1)^{2} \cdot 3 \cdot (2 - 4) + (-1)^{3} \cdot 0 \cdot (2 - 0) + (-1)^{4} \cdot (-1) \cdot (4 - 0)
  \\
  &= 3 \cdot (-2) + 0 + (-1) \cdot 4\\
  &= -10
\end{alignat*}

\newpage
(b)

\begin{alignat*}{2}
  \begin{vmatrix}
      3 &  0 &  -1\\
      1 & 1  & 1\\
      0 &  4 &  2
  \end{vmatrix}
  &=
  -\begin{vmatrix}
      1 & 1  & 1\\
      3 &  0 &  -1\\
      0 &  4 &  2
  \end{vmatrix}
  \\
  &=
  \begin{vmatrix}
      1 & 1  & 1\\
      0 &  4 &  2\\
      3 &  0 &  -1
  \end{vmatrix}
  \\
  &=
  \begin{vmatrix}
      1 & 1  & 1\\
      0 &  4 &  2\\
      0 &  0 &  -\frac{5}{2}
  \end{vmatrix}
\end{alignat*}

Logo, o determinante dessa matriz é $1 \cdot 4 \cdot \left( -\frac{5}{2} \right) = -10$.\\

\textbf{Questão 19, pág. 117:}\\

Calculando as menores:

\begin{alignat*}{2}
  M_{11} &=
  \begin{vmatrix}
    2  & -1\\
    1 &  1
  \end{vmatrix}
  = 2 - (-1) = 3 \\
  M_{12} &=
  \begin{vmatrix}
    0  & -1\\
    -3 &  1
  \end{vmatrix}
  = 0 - 3 = -3 \\
  M_{13} &=
  \begin{vmatrix}
    0  & 2\\
    -3 &  1
  \end{vmatrix}
  = 0 - (-6) = 6 \\
  M_{21} &= 
  \begin{vmatrix}
    5  & 2\\
    1 &  1
  \end{vmatrix}
  = 5 - 2 = 3 \\
  M_{22} &= 
  \begin{vmatrix}
    -1  & 2\\
    -3 &  1
  \end{vmatrix}
  = -1 - (-6) = 5 \\
  M_{23} &=
  \begin{vmatrix}
    -1  & 5\\
    -3 &  1
  \end{vmatrix}
  = -1 - (-15) = 14 \\
  M_{31} &=
  \begin{vmatrix}
    5  & 2\\
    2 &  -1
  \end{vmatrix}
  = -5 - 4 = -9 \\
  M_{32} &= 
  \begin{vmatrix}
    -1  & 2\\
    0 &  -1
  \end{vmatrix}
  = 1 - 0 = 1 \\
  M_{32} &= 
  \begin{vmatrix}
    -1  & 5\\
    0 &  2
  \end{vmatrix}
  = -2 - 0 = -2 \\
\end{alignat*}

Calculando os cofatores:

\begin{alignat*}{2}
  C_{11} &= (-1)^{1+1} \cdot M_{11} = 1 \cdot 3 = 3\\
  C_{12} &= (-1)^{1+2} \cdot M_{12} = (-1) \cdot (-3) = 3\\
  C_{13} &= (-1)^{1+3} \cdot M_{12} = 1 \cdot 6 = 6\\
  C_{21} &= (-1)^{2+1} \cdot M_{21} = (-1) \cdot 3 = -3\\
  C_{22} &= (-1)^{2+2} \cdot M_{22} = 1 \cdot 5 = 5\\
  C_{23} &= (-1)^{2+3} \cdot M_{23} = (-1) \cdot 14 = -14\\
  C_{31} &= (-1)^{3+1} \cdot M_{31} = 1 \cdot (-9) = -9\\
  C_{32} &= (-1)^{3+2} \cdot M_{32} = (-1) \cdot 1 = -1\\
  C_{33} &= (-1)^{3+3} \cdot M_{33} = 1 \cdot (-2) = -2
\end{alignat*}

\newpage

A matriz adjunta será, então, a transposta da matriz formada pelos cofatores calculados anteriormente:

\begin{align*}
  adj(M3) = 
  \begin{bmatrix}
    3 & -3 & -9\\
    3 & 5 & -1\\
    6 & -14 & -2
  \end{bmatrix}  
\end{align*}

Encontramos a inversa da matriz M3, multiplicando essa matriz adjunta pelo inveso do determinante calculado no exercício 3. Sendo assim:

\begin{align*}
  M3^{-1} = \frac{1}{24} \cdot
  \begin{bmatrix}
    3 & -3 & -9\\
    3 & 5 & -1\\
    6 & -14 & -2
  \end{bmatrix}  
  =
  \begin{bmatrix}
    \frac{1}{8} & -\frac{1}{8} & -\frac{3}{8}\\
    \frac{1}{8} & \frac{5}{24} & -\frac{1}{24}\\
    \frac{1}{4} & -\frac{7}{12} & -\frac{1}{12}
  \end{bmatrix}  
\end{align*}

\textbf{Questão 21, pág. 117:}\\

Calculando as menores:
\begin{alignat*}{2}
  M_{11} &=
  \begin{vmatrix}
    1 & 1\\
    4 & 2
  \end{vmatrix}
  = 2 - 4 = -2 \\
  M_{12} &=
  \begin{vmatrix}
    1 & 1\\
    0 & 2
  \end{vmatrix}
  = 2 - 0 = 2 \\
  M_{13} &=
  \begin{vmatrix}
    1 & 1\\
    0 & 4
  \end{vmatrix}
  = 4 - 0 = 4 \\
  M_{21} &=
  \begin{vmatrix}
    0 & -1\\
    4 & 2
  \end{vmatrix}
  = 0 - (-4) = 4 \\
  M_{22} &=
  \begin{vmatrix}
    3 & -1\\
    0 & 2
  \end{vmatrix}
  = 6 - 0 = 6 \\
  M_{23} &=
  \begin{vmatrix}
    3 & 0\\
    0 & 4
  \end{vmatrix}
  = 12 - 0 = 12 \\
  M_{31} &=
  \begin{vmatrix}
    0 & -1\\
    1 & 1
  \end{vmatrix}
  = 0 - (-1) = 1 \\
  M_{32} &=
  \begin{vmatrix}
    3 & -1\\
    1 & 1
  \end{vmatrix}
  = 3 - (-1) = 4 \\
  M_{33} &=
  \begin{vmatrix}
    3 & 0\\
    1 & 1
  \end{vmatrix}
  = 3 - 0 = 3 \\
\end{alignat*}

Calculando os cofatores:

\begin{alignat*}{2}
  C_{11} &= (-1)^{1+1} \cdot M_{11} = 1 \cdot (-2) = -2\\
  C_{12} &= (-1)^{1+2} \cdot M_{12} = (-1) \cdot 2 = -2\\
  C_{13} &= (-1)^{1+3} \cdot M_{12} = 1 \cdot 4 = 4\\
  C_{21} &= (-1)^{2+1} \cdot M_{21} = (-1) \cdot 4 = -4\\
  C_{22} &= (-1)^{2+2} \cdot M_{22} = 1 \cdot 6 = 6\\
  C_{23} &= (-1)^{2+3} \cdot M_{23} = (-1) \cdot 12 = -12\\
  C_{31} &= (-1)^{3+1} \cdot M_{31} = 1 \cdot 1 = 1\\
  C_{32} &= (-1)^{3+2} \cdot M_{32} = (-1) \cdot 4 = -4\\
  C_{33} &= (-1)^{3+3} \cdot M_{33} = 1 \cdot 3 = 3
\end{alignat*}

\newpage
A matriz adjunta será, então, a transposta da matriz formada pelos cofatores calculados anteriormente:

\begin{align*}
  adj(M5) = 
  \begin{bmatrix}
    -2 & -4 & 1\\
    -2 & 6 & -4\\
    4 & -12 & 3
  \end{bmatrix}  
\end{align*}

Encontramos a inversa da matriz M3, multiplicando essa matriz adjunta pelo inveso do determinante calculado no exercício 3. Sendo assim:

\begin{align*}
  M5^{-1} = -\frac{1}{10} \cdot
  \begin{bmatrix}
    -2 & -4 & 1\\
    -2 & 6 & -4\\
    4 & -12 & 3
  \end{bmatrix}  
  =
  \begin{bmatrix}
    \frac{1}{5} & \frac{2}{5} & -\frac{1}{10}\\
    \frac{1}{5} & -\frac{3}{5} & \frac{2}{5}\\
    -\frac{2}{5} & \frac{6}{5} & -\frac{3}{10}
  \end{bmatrix}  
\end{align*}

\end{document}