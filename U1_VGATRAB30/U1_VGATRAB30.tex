
\documentclass[a4paper,12pt]{article}
\author{}
\date{}
\usepackage[papersize={216mm,330mm},tmargin=20mm,bmargin=20mm,lmargin=20mm,rmargin=20mm]{geometry}
\usepackage[brazil]{babel}
\usepackage[utf8]{inputenc}
\usepackage{amsmath,amssymb} %,mathabx}%\for eqref
\usepackage{lscape}
\usepackage{graphicx}
\usepackage[colorinlistoftodos]{todonotes}
\usepackage{fancyhdr}
\usepackage{tasks}
\usepackage{float}
\usepackage{multicol}
\usepackage{color}
\usepackage{ragged2e} % justifying
\usepackage{colortbl}
\usepackage{cancel}
\usepackage{stackengine}
\usepackage{mathtools}
\usepackage{tikz}   %create images
\usepackage{tkz-fct}    %create graphics

%====================================================
%============= INFORMACAO DE ENTRADA=================
%====================================================

\def\nomedoaluno{Lucas Manoel Arruda Pinheiro}
\def\matricula{20230046509}

%IMD1002 - Análise Combinatória
%IMD0024 - Cálculo 1
%IMD0034 - Vetores e Geometria Analítica
\def\coddisciplina{IMD0034}
\def\nomedisciplina{VETORES E GEOMETRIA ANALÍTICA}
\def\codturma{01}
\def\codatividade{U1\_VGATRAB30}

%====================================================
%====================== FIM =========================
%====================================================

\pagestyle{fancy}
\fancyhf{}
\lhead{Aluno: \nomedoaluno\\ Matrícula: \matricula}
\chead{\thepage}
\rhead{Instituto Metrópole Digital - UFRN \\
Turma \codturma}
\lfoot{\nomedisciplina}
\cfoot{Atividade \codatividade}
\rfoot{Prof. Samyr Jácome}

\title{
\vspace{-2cm}
\Large \textbf{Universidade Federal do Rio Grande do Norte}\\ 
Instituto Metrópole Digital \\ 
\coddisciplina $\;$ - \nomedisciplina \\ 
TURMA \codturma \\
\vspace{5mm} \Large\textbf{ATIVIDADE \codatividade} \\
\normalsize Natal-RN, \today\\
\vspace{0.7cm} \large \textit{Prof. Samyr Jácome}\\

\justifying
\vspace{0.5cm} \hspace{-0.82cm}
\begin{minipage}{.6\linewidth}
    \large \textbf{Aluno:} \nomedoaluno
\end{minipage}
\begin{minipage}{.4\linewidth}
    \begin{flushright}
        \large \textbf{Matrícula:} \matricula
    \end{flushright}
\end{minipage}
}

% Tirar a identação do paragrafo no texto
\def\tirarident{\setlength{\parindent}{0cm}} % padrão 15pt.

\setlength{\marginparwidth}{2cm}

%---------------------------------------------------------------
%---------------COMANDO PARA SETA DE ESCALONAMENTO--------------
%---------------------------------------------------------------

\newcommand{\seta}[3][-0.05cm]{%
  \stackon[#1]{
    $\xrightarrow{\mathmakebox[\setawidth]{}}$}{
    \scriptsize{$#2 \,\leftarrow\, #3$}
    }
}

\newcommand{\troca}[3][-0.05cm]{%
  \stackon[#1]{
    $\xrightarrow{\mathmakebox[\setawidth]{}}$}{
    \scriptsize{$#2 \,\leftrightarrow\, #3$}
    }
}

\newlength{\setawidth}% row operation width
\AtBeginDocument{\setlength{\setawidth}{2.0cm}}

%---------------------------------------------------------------
%--------------------------FIM----------------------------------
%---------------------------------------------------------------

\begin{document}
\maketitle

\vspace{-2cm}
\section*{Exercícios}

\tirarident

\textbf{Questão 13, pág. 59:}\\

\begin{alignat*}{2}
    \begin{bmatrix}
         3 &  4 & -1 & 1 & 0 & 0\\
         1 &  0 & 3 & 0 & 1 & 0\\
         2 &  5 & -4 & 0 & 0 & 1
    \end{bmatrix}
    &\!\begin{aligned}
        &&\troca{L_1}{L_2}
    \end{aligned}
    \begin{bmatrix}
        1 &  0 & 3 & 0 & 1 & 0\\
        3 &  4 & -1 & 1 & 0 & 0\\
        2 &  5 & -4 & 0 & 0 & 1
    \end{bmatrix}
    \\
    &\!\begin{aligned}
        &\seta{L_2}{L_2 - 3L_1}\\
        &\seta{L_3}{L_3 - 2L_1}
    \end{aligned}
    \begin{bmatrix}
        1 &  0 & 3 & 0 & 1 & 0\\
        0 &  4 & -10 & 1 & -3 & 0\\
        0 &  5 & -10 & 0 & -2 & 1
    \end{bmatrix}
    \\
    &\!\begin{aligned}
        &\seta{L_2}{L_2/2}\\
        &\seta{L_3}{L_3/5}
    \end{aligned}
    \begin{bmatrix}
        1 &  0 & 3 & 0 & 1 & 0\\
        0 &  2 & -5 & \frac{1}{2} & -\frac{3}{2} & 0\\
        0 &  1 & -2 & 0 & -\frac{2}{5} & \frac{1}{5}
    \end{bmatrix}
    \\
    &\!\begin{aligned}
        &\seta{L_2}{L_2 - 2L_3}
    \end{aligned}
    \begin{bmatrix}
        1 &  0 & 3 & 0 & 1 & 0\\
        0 &  0 & -1 & \frac{1}{2} & -\frac{7}{10} & -\frac{2}{5}\\
        0 &  1 & -2 & 0 & -\frac{2}{5} & \frac{1}{5}
    \end{bmatrix}
    \\
    &\!\begin{aligned}
        &\seta{L_2}{-1 \cdot L_2}
    \end{aligned}
    \begin{bmatrix}
        1 &  0 & 3 & 0 & 1 & 0\\
        0 &  0 & 1 & -\frac{1}{2} & \frac{7}{10} & \frac{2}{5}\\
        0 &  1 & -2 & 0 & -\frac{2}{5} & \frac{1}{5}
    \end{bmatrix}
    \\
    &\!\begin{aligned}
        &\seta{L_1}{L_1 - 3L_2}\\
        &\seta{L_3}{L_3 + 2L_2}
    \end{aligned}
    \begin{bmatrix}
        1 &  0 & 0 & \frac{3}{2} & -\frac{11}{10} & -\frac{6}{5}\\
        0 &  0 & 1 & -\frac{1}{2} & \frac{7}{10} & \frac{2}{5}\\
        0 &  1 & 0 & -1 & 1 & 1
    \end{bmatrix}
    \\
    &\!\begin{aligned}
        &\troca{L_2}{L_3}
    \end{aligned}
    \begin{bmatrix}
        1 &  0 & 0 & \frac{3}{2} & -\frac{11}{10} & -\frac{6}{5}\\
        0 &  1 & 0 & -1 & 1 & 1\\
        0 &  0 & 1 & -\frac{1}{2} & \frac{7}{10} & \frac{2}{5}
    \end{bmatrix}
\end{alignat*}

Logo, a inversa dessa matriz é

\begin{alignat*}{2}
    \begin{bmatrix}
        \frac{3}{2} & -\frac{11}{10} & -\frac{6}{5}\\
        -1 & 1 & 1\\
        -\frac{1}{2} & \frac{7}{10} & \frac{2}{5}   
    \end{bmatrix}
\end{alignat*}    

\textbf{Questão 14, pág. 59:}\\

\begin{alignat*}{2}
    \begin{bmatrix}
         1 &  2 & 0 & 1 & 0 & 0\\
         2 &  1 & 2 & 0 & 1 & 0\\
         0 &  2 & 1 & 0 & 0 & 1
    \end{bmatrix}
    &\!\begin{aligned}
        &&\seta{L_2}{L_2 - 2L_1}
    \end{aligned}
    \begin{bmatrix}
         1 &  2 & 0 & 1 & 0 & 0\\
         0 &  -3 & 2 & -2 & 1 & 0\\
         0 &  2 & 1 & 0 & 0 & 1
    \end{bmatrix}
    \\
    &\!\begin{aligned}
        &&\seta{L_2}{L_2 - 2L_3}
    \end{aligned}
    \begin{bmatrix}
         1 &  2 & 0 & 1 & 0 & 0\\
         0 &  -7 & 0 & -2 & 1 & -2\\
         0 &  2 & 1 & 0 & 0 & 1
    \end{bmatrix}
    \\
    &\!\begin{aligned}
        &&\seta{L_2}{-\frac{1}{7}L_2}
    \end{aligned}
    \begin{bmatrix}
         1 &  2 & 0 & 1 & 0 & 0\\
         0 &  1 & 0 & \frac{2}{7} & -\frac{1}{7} & \frac{2}{7}\\
         0 &  2 & 1 & 0 & 0 & 1
    \end{bmatrix}
    \\
    &\!\begin{aligned}
        &&\seta{L_1}{L_1 - 2L_2}\\
        &&\seta{L_3}{L_3 - 2L_2}
    \end{aligned}
    \begin{bmatrix}
         1 &  0 & 0 & \frac{3}{7} & \frac{2}{7} & -\frac{4}{7}\\
         0 &  1 & 0 & \frac{2}{7} & -\frac{1}{7} & \frac{2}{7}\\
         0 &  0 & 1 & -\frac{4}{7} & \frac{2}{7} & \frac{3}{7}
    \end{bmatrix}
\end{alignat*}

Logo, a inversa dessa matriz é

\begin{alignat*}{2}
    \begin{bmatrix}
        \frac{3}{7} & \frac{2}{7} & -\frac{4}{7}\\
        \frac{2}{7} & -\frac{1}{7} & \frac{2}{7}\\
        -\frac{4}{7} & \frac{2}{7} & \frac{3}{7} 
    \end{bmatrix}
\end{alignat*}  

\textbf{Questão 15, pág. 59:}\\

\begin{alignat*}{2}
    \begin{bmatrix}
         -1 &  3 & -4 & 1 & 0 & 0\\
         2 &  4 & 1 & 0 & 1 & 0\\
         -4 &  2 & -9 & 0 & 0 & 1
    \end{bmatrix}
    &\!\begin{aligned}
        &&\seta{L_1}{L_1 + 4L_2}\\
        &&\seta{L_3}{L_3 + 9L_2}
    \end{aligned}
    \begin{bmatrix}
        7 &  19 & 0 & 1 & 4 & 0\\
        2 &  4 & 1 & 0 & 1 & 0\\
        14 &  38 & 0 & 0 & 9 & 1
   \end{bmatrix}
   \\
   &\!\begin{aligned}
        &&\seta{L_3}{L_3 - 2L_1}
    \end{aligned}
    \begin{bmatrix}
        7 &  19 & 0 & 1 & 4 & 0\\
        2 &  4 & 1 & 0 & 1 & 0\\
        0 &  0 & 0 & -2 & 1 & 1
   \end{bmatrix}
\end{alignat*}

Não podemos continuar com o escalonamento, logo essa matriz não tem inversa.\\

\textbf{Questão 3, pág. 65:}\\

Sabendo que a representação matricial desse sistema é:

\begin{align*}
    \underbrace{
    \begin{bmatrix}
        1 & 3 & 1\\
        2  & 2 & 1\\
        2 & 3 & 1
    \end{bmatrix}}_{A} \cdot
    \underbrace{
    \begin{bmatrix}
        x_1\\
        x_2\\
        x_3
    \end{bmatrix}}_{x}
    =
    \underbrace{
    \begin{bmatrix}
        4\\
        -1\\
        3
    \end{bmatrix}}_{B}
\end{align*}

Podemos afirmar que $x = A^{-1}B$, pelo teorema 1.6.2. Para encontrar $A^{-1}$:

\begin{alignat*}{2}
    \begin{bmatrix}
         1 &  3 & 1 & 1 & 0 & 0\\
         2 &  2 & 1 & 0 & 1 & 0\\
         2 &  3 & 1 & 0 & 0 & 1
    \end{bmatrix}
    &\!\begin{aligned}
        &&\seta{L_2}{L_2 - 2L_1}\\
        &&\seta{L_3}{L_3 - 2L_2}
    \end{aligned}
    \begin{bmatrix}
        1 &  3 & 1 & 1 & 0 & 0\\
        0 &  -4 & -1 & -2 & 1 & 0\\
        0 &  -3 & -1 & -2 & 0 & 1
    \end{bmatrix}
    \\
    &\!\begin{aligned}
        &&\seta{L_2}{-L_2}\\
        &&\seta{L_3}{-L_3}
    \end{aligned}
    \begin{bmatrix}
        1 &  3 & 1 & 1 & 0 & 0\\
        0 &  4 & 1 & 2 & -1 & 0\\
        0 &  3 & 1 & 2 & 0 & -1
    \end{bmatrix}
    \\
    &\!\begin{aligned}
        &&\seta{L_1}{L_1 -L_3}\\
        &&\seta{L_2}{L_2 -L_3}
    \end{aligned}
    \begin{bmatrix}
        1 &  0 & 0 & -1 & 0 & 1\\
        0 &  1 & 0 & 0 & -1 & 1\\
        0 &  3 & 1 & 2 & 0 & -1
    \end{bmatrix}
    \\
    &\!\begin{aligned}
        &&\seta{L_3}{L_3 - 3L_2}\\
    \end{aligned}
    \begin{bmatrix}
        1 &  0 & 0 & -1 & 0 & 1\\
        0 &  1 & 0 & 0 & -1 & 1\\
        0 &  0 & 1 & 2 & 3 & -4
    \end{bmatrix}
\end{alignat*}

Logo, a inversa da matriz A é:

\begin{alignat*}{2}
    \begin{bmatrix}
        -1 & 0 & 1\\
        0 & -1 & 1\\
        2 & 3 & -4
    \end{bmatrix}
\end{alignat*}  

e, assim, podemos encontrar o conjunto solução do sistema através da equação:

\begin{align*}
    \underbrace{
    \begin{bmatrix}
        x_1\\
        x_2\\
        x_3
    \end{bmatrix}}_{x}
    &=
    \underbrace{
    \begin{bmatrix}
        -1 & 0 & 1\\
        0 & -1 & 1\\
        2 & 3 & -4
    \end{bmatrix}}_{A^{-1}} \cdot
    \underbrace{
    \begin{bmatrix}
        4\\
        -1\\
        3
    \end{bmatrix}}_{B}\\
    &=
    \begin{bmatrix}
        (-1)\cdot4 + 0\cdot(-1) + 1\cdot3\\
        0\cdot4 + -1\cdot(-1) + 1\cdot3\\
        2\cdot4 + 3\cdot(-1) + (-4)\cdot3
    \end{bmatrix}\\
    &=
    \begin{bmatrix}
        -1\\
        4\\
        -7
    \end{bmatrix}
\end{align*}

Sendo assim, o conjunto solução desse sistema é $\{-1, 4, -7\}$\\

\textbf{Questão 5, pág. 65:}\\

Sabendo que a representação matricial desse sistema é:

\begin{align*}
    \underbrace{
    \begin{bmatrix}
        1 & 1 & 1\\
        1  & 1 & -4\\
        -4 & 1 & 1
    \end{bmatrix}}_{A} \cdot
    \underbrace{
    \begin{bmatrix}
        x_1\\
        x_2\\
        x_3
    \end{bmatrix}}_{x}
    =
    \underbrace{
    \begin{bmatrix}
        5\\
        10\\
        0
    \end{bmatrix}}_{B}
\end{align*}

Podemos afirmar que $x = A^{-1}B$, pelo teorema 1.6.2. Para encontrar $A^{-1}$:

\begin{alignat*}{2}
    \begin{bmatrix}
         1 &  1 &  1 & 1 & 0 & 0\\
         1 & 1  & -4 & 0 & 1 & 0\\
        -4 &  1 &  1 & 0 & 0 & 1
    \end{bmatrix}
    &\!\begin{aligned}
        &&\seta{L_2}{L_2 - L_1}\\
        &&\seta{L_3}{L_3 - L_1}
    \end{aligned}
    \begin{bmatrix}
        1 &  1 &  1 & 1 & 0 & 0\\
        0 & 0  & -5 & -1 & 1 & 0\\
        -5 &  0 &  0 & -1 & 0 & 1
    \end{bmatrix}
    \\
    &\!\begin{aligned}
        &&\seta{L_2}{-\frac{1}{5}L_2}\\
        &&\seta{L_3}{-\frac{1}{5}L_3}
    \end{aligned}
    \begin{bmatrix}
        1 &  1 &  1 & 1 & 0 & 0\\
        0 & 0  & 1 & \frac{1}{5} & -\frac{1}{5} & 0\\
        1 &  0 &  0 & \frac{1}{5} & 0 & -\frac{1}{5}
    \end{bmatrix}
    \\
    &\!\begin{aligned}
        &&\seta{L_1}{L_1 - L_2}
    \end{aligned}
    \begin{bmatrix}
        1 &  1 &  0 & \frac{4}{5} & \frac{1}{5} & 0\\
        0 & 0  & 1 & \frac{1}{5} & -\frac{1}{5} & 0\\
        1 &  0 &  0 & \frac{1}{5} & 0 & -\frac{1}{5}
    \end{bmatrix}
    \\
    &\!\begin{aligned}
        &&\seta{L_1}{L_1 - L_3}
    \end{aligned}
    \begin{bmatrix}
        0 &  1 &  0 & \frac{3}{5} & \frac{1}{5} & \frac{1}{5}\\
        0 & 0  & 1 & \frac{1}{5} & -\frac{1}{5} & 0\\
        1 &  0 &  0 & \frac{1}{5} & 0 & -\frac{1}{5}
    \end{bmatrix}
    \\
    &\!\begin{aligned}
        &&\troca{L_1}{L_2}
    \end{aligned}
    \begin{bmatrix}
        0 & 0  & 1 & \frac{1}{5} & -\frac{1}{5} & 0\\
        0 &  1 &  0 & \frac{3}{5} & \frac{1}{5} & \frac{1}{5}\\
        1 &  0 &  0 & \frac{1}{5} & 0 & -\frac{1}{5}
    \end{bmatrix}
    \\
    &\!\begin{aligned}
        &&\troca{L_1}{L_3}
    \end{aligned}
    \begin{bmatrix}
        1 &  0 &  0 & \frac{1}{5} & 0 & -\frac{1}{5}\\
        0 &  1 &  0 & \frac{3}{5} & \frac{1}{5} & \frac{1}{5}\\
        0 & 0  & 1 & \frac{1}{5} & -\frac{1}{5} & 0
    \end{bmatrix}
\end{alignat*}

Logo, a inversa da matriz A é:

\begin{alignat*}{2}
    \begin{bmatrix}
        \frac{1}{5} & 0 & -\frac{1}{5}\\
        \frac{3}{5} & \frac{1}{5} & \frac{1}{5}\\
        \frac{1}{5} & -\frac{1}{5} & 0
    \end{bmatrix}
\end{alignat*}  

e, assim, podemos encontrar o conjunto solução do sistema através da equação:

\begin{align*}
    \underbrace{
    \begin{bmatrix}
        x_1\\
        x_2\\
        x_3
    \end{bmatrix}}_{x}
    &=
    \underbrace{
    \begin{bmatrix}
        \frac{1}{5} & 0 & -\frac{1}{5}\\
        \frac{3}{5} & \frac{1}{5} & \frac{1}{5}\\
        \frac{1}{5} & -\frac{1}{5} & 0
    \end{bmatrix}}_{A^{-1}} \cdot
    \underbrace{
    \begin{bmatrix}
        5\\
        10\\
        0
    \end{bmatrix}}_{B}\\
    &= 
    \begin{bmatrix}
        5\cdot \frac{1}{5} + 10\cdot0 + 0\cdot-\frac{1}{5}\\
        5\cdot\frac{3}{5} + 10\cdot\frac{1}{5} + 0\cdot\frac{1}{5}\\
        5\cdot\frac{1}{5} + 10\cdot-\frac{1}{5} + 0\cdot0
    \end{bmatrix}\\
    &=
    \begin{bmatrix}
        1\\
        5\\
        -1
    \end{bmatrix}
\end{align*}

Sendo assim, o conjunto solução desse sistema é $\{1, 5, -1\}$

\end{document}