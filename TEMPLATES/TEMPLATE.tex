
\documentclass[a4paper,12pt]{article}
\author{}
\date{}
\usepackage[papersize={216mm,330mm},tmargin=20mm,bmargin=20mm,lmargin=20mm,rmargin=20mm]{geometry}
\usepackage[brazil]{babel}
\usepackage[utf8]{inputenc}
\usepackage{amsmath,amssymb} %,mathabx}%\for eqref
\usepackage{lscape}
\usepackage{graphicx}
\usepackage[colorinlistoftodos]{todonotes}
\usepackage{fancyhdr}
\usepackage{tasks}
\usepackage{float}
\usepackage{multicol}
\usepackage{color}
\usepackage{ragged2e} % justifying
\usepackage{colortbl}
\usepackage{cancel}
\usepackage{stackengine}
\usepackage{mathtools}
\usepackage{tikz}   %create images
\usepackage{tkz-fct}    %create graphics

%====================================================
%============= INFORMACAO DE ENTRADA=================
%====================================================

\def\nomedoaluno{Lucas Manoel Arruda Pinheiro}
\def\matricula{20230046509}

%IMD1002 - Análise Combinatória
%IMD0024 - Cálculo 1
%IMD0034 - Vetores e Geometria Analítica
\def\coddisciplina{IMD0034}
\def\nomedisciplina{VETORES E GEOMETRIA ANALÍTICA}
\def\codturma{01}
\def\codatividade{U1\_VGATRABXX}

%====================================================
%====================== FIM =========================
%====================================================

\pagestyle{fancy}
\fancyhf{}
\lhead{Aluno: \nomedoaluno\\ Matrícula: \matricula}
\chead{\thepage}
\rhead{Instituto Metrópole Digital - UFRN \\
Turma \codturma}
\lfoot{\nomedisciplina}
\cfoot{Atividade \codatividade}
\rfoot{Prof. Samyr Jácome}

\title{
\vspace{-2cm}
\Large \textbf{Universidade Federal do Rio Grande do Norte}\\ 
Instituto Metrópole Digital \\ 
\coddisciplina $\;$ - \nomedisciplina \\ 
TURMA \codturma \\
\vspace{5mm} \Large\textbf{ATIVIDADE \codatividade} \\
\normalsize Natal-RN, \today\\
\vspace{0.7cm} \large \textit{Prof. Samyr Jácome}\\

\justifying
\vspace{0.5cm} \hspace{-0.82cm}
\begin{minipage}{.6\linewidth}
    \large \textbf{Aluno:} \nomedoaluno
\end{minipage}
\begin{minipage}{.4\linewidth}
    \begin{flushright}
        \large \textbf{Matrícula:} \matricula
    \end{flushright}
\end{minipage}
}

% Tirar a identação do paragrafo no texto
\def\tirarident{\setlength{\parindent}{0cm}} % padrão 15pt.

\setlength{\marginparwidth}{2cm}

%---------------------------------------------------------------
%---------------COMANDO PARA SETA DE ESCALONAMENTO--------------
%---------------------------------------------------------------

\newcommand{\seta}[3][-0.05cm]{%
  \stackon[#1]{
    $\xrightarrow{\mathmakebox[\setawidth]{}}$}{
    \scriptsize{$#2 \,\leftarrow\, #3$}
    }
}

\newcommand{\troca}[3][-0.05cm]{%
  \stackon[#1]{
    $\xrightarrow{\mathmakebox[\setawidth]{}}$}{
    \scriptsize{$#2 \,\leftrightarrow\, #3$}
    }
}

\newlength{\setawidth}% row operation width
\AtBeginDocument{\setlength{\setawidth}{2.0cm}}

%---------------------------------------------------------------
%--------------------------FIM----------------------------------
%---------------------------------------------------------------

\begin{document}
\maketitle

\vspace{-2cm}
\section*{Exercícios}

\tirarident

\textbf{Questão 01, pág. xx:}\\

Comando Latex para fazer escalonamento de matrizes\\

\begin{alignat*}{2}
    %------ PRIMEIRA LINHA -------
    \begin{bmatrix}
         1 &  2 & 0\\
        -1 &  1 & 2\\
         1 &  0 & 2
    \end{bmatrix}
    &\!\begin{aligned} % O \begin{aligned} \end{aligned} deve ser usado sempre que o houver mais de suas setas para idicar as operações de escalonamento.
        &\seta{L_2}{L_2 + L_1}\\
        &\seta{L_3}{L_3 - L_1}
    \end{aligned}
    \begin{bmatrix}
         1 &  2 & 0\\
         0 &  3 & 2\\
         0 & -2 & 2
    \end{bmatrix}
    % deve-se usa && quando quiser escrever a próxima matriz na mesma linha
    &&\troca{L_2}{-\frac{1}{2}L_3}
    \begin{bmatrix}
         1 &  2 & 0\\
         0 &  1 & -1\\
         0 & 3 & 2
    \end{bmatrix}
    %------ FIM DAPRIMEIRA LINHA -------
    \\ % PULA LINHA
    %------ INICIO DA SEGUNDA LINHA -------
    &\!\begin{aligned} 
        &\seta{L_1}{L_1 - 2L_2}\\
        &\seta{L_3}{L_3 - 3L_2}
    \end{aligned}
    \begin{bmatrix}
         1 & 0 & 2\\
         0 & 1 & -1\\
         0 & 0 & 4
    \end{bmatrix}
    &&\seta{L_3}{\frac{1}{4}L_3}
    \begin{bmatrix}
         1 & 0 & 2\\
         0 & 1 & -1\\
         0 & 0 & 1
    \end{bmatrix}
    %------ FIM DA SEGUNDA LINHA -------
    \\ % PULA LINHA
    %------ INICIO DA TERCEIRA LINHA -------
    &\!\begin{aligned} 
        &\seta{L_1}{L_1 - 2L_3}\\
        &\seta{L_2}{L_2 + L_3}
    \end{aligned}
    \begin{bmatrix}
         1 & 0 & 0\\
         0 & 1 & 0\\
         0 & 0 & 1
    \end{bmatrix} % ATENCAO: NAO DEIXAR LINHA EM BRANCO ENTRE \end{bmatrix} E \end{alignat*}
\end{alignat*}


\textbf{Questão 02, pág. xx:}\\

A adição da matriz $A$ com a matriz $B$ é

    \begin{align*}
        A + B &=
            \underbrace{
            \begin{bmatrix}
                10 & 7 & 4\\
                9  & 8 & 6\\
                11 & 8 & 6\\
                10 & 9 & 5
            \end{bmatrix}}_{A} +
            \underbrace{
            \begin{bmatrix}
                11 & 10 & 6\\
                10 & \phantom{2}9  & 6\\
                10 & \phantom{2}8  & 4\\
                \phantom{2}8  & \phantom{2}7  & 4\\
            \end{bmatrix}}_{B} 
        =
        \begin{bmatrix}
                10+11 & 7+11 & 4+6\\
                 \phantom{2}9+10  & 8+\phantom{2}9 & 6+6\\
                11+10 & 8+\phantom{2}8 & 6+4\\
                10+\phantom{2}8 & 9+\phantom{2}7 & 5+4
        \end{bmatrix}\\
        & =
        \underbrace{\begin{bmatrix}
            21 & 19 & 11\\
            21 & 17 & 12\\
            20 & 16 & \phantom{2}8\\
            17 & 14 & \phantom{2}9\\
        \end{bmatrix}}_{A + B}
    \end{align*}

\textbf{Questão 03, pág. xx:}\\

A multiplicação de uma matriz por uma matriz coluna gera uma matriz coluna

    \vspace{0.5cm} 
    \centering
$
        \begin{bmatrix}
          x \\ y \\ z \\ w
        \end{bmatrix}
        =
        \begin{bmatrix}
            11.5 & 9.5 & 5.5\\
            11.5 & 8.5 & 6.0\\
            10.0 & 8.0 & 4.0\\
            \phantom{0}8.5 & 7.0 & 4.5\\
        \end{bmatrix}
        \begin{bmatrix}
            2 \\ 4 \\ 9\\
        \end{bmatrix}        
        = 
        \begin{bmatrix}
            11.5\cdot 2 + 9.5\cdot 4 + 5.5\cdot 9\\
            11.5\cdot 2 + 8.5\cdot 4 + 6.0\cdot 9\\
            10.0\cdot 2 + 8.0\cdot 4 + 4.0\cdot 9\\
            8.5\cdot 2 + 7.0\cdot 4 + 4.5\cdot 9\\
        \end{bmatrix}
        =
        \begin{bmatrix}
            110.5 \\ 111 \\ 88 \\85.5
        \end{bmatrix}        
$
    

\end{document}