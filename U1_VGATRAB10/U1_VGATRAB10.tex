\documentclass[a4paper,12pt]{article}
\author{}
\date{}
\usepackage[papersize={216mm,330mm},tmargin=20mm,bmargin=20mm,lmargin=20mm,rmargin=20mm]{geometry}
\usepackage[brazil]{babel}
\usepackage[utf8]{inputenc}
\usepackage{amsmath,amssymb} %,mathabx}%\for eqref
\usepackage{lscape}
\usepackage{graphicx}
\usepackage[colorinlistoftodos]{todonotes}
\usepackage{fancyhdr}
\usepackage{tasks}
\usepackage{float}
\usepackage{multicol}
\usepackage{color}
\usepackage{ragged2e} % justifying
\usepackage{colortbl}
\usepackage{cancel}
\usepackage{stackengine}
\usepackage{mathtools}
\usepackage{tikz}   %create images
\usepackage{tkz-fct}    %create graphics

%====================================================
%============= INFORMACAO DE ENTRADA=================
%====================================================

\def\nomedoaluno{Lucas Manoel Arruda Pinheiro}
\def\matricula{20230046509}

%IMD1002 - Análise Combinatória
%IMD0024 - Cálculo 1
%IMD0034 - Vetores e Geometria Analítica
\def\coddisciplina{IMD0034}
\def\nomedisciplina{VETORES E GEOMETRIA ANALÍTICA}
\def\codturma{01}
\def\codatividade{U1\_VGATRAB10}

%====================================================
%====================== FIM =========================
%====================================================

\pagestyle{fancy}
\fancyhf{}
\lhead{Aluno: \nomedoaluno\\ Matrícula: \matricula}
\chead{\thepage}
\rhead{Instituto Metrópole Digital - UFRN \\
Turma \codturma}
\lfoot{\nomedisciplina}
\cfoot{Atividade \codatividade}
\rfoot{Prof. Samyr Jácome}

\title{
\vspace{-2cm}
\Large \textbf{Universidade Federal do Rio Grande do Norte}\\ 
Instituto Metrópole Digital \\ 
\coddisciplina $\;$ - \nomedisciplina \\ 
TURMA \codturma \\
\vspace{5mm} \Large\textbf{ATIVIDADE \codatividade} \\
\normalsize Natal-RN, \today\\
\vspace{0.7cm} \large \textit{Prof. Samyr Jácome}\\

\justifying
\vspace{0.5cm} \hspace{-0.82cm}
\begin{minipage}{.6\linewidth}
    \large \textbf{Aluno:} \nomedoaluno
\end{minipage}
\begin{minipage}{.4\linewidth}
    \begin{flushright}
        \large \textbf{Matrícula:} \matricula
    \end{flushright}
\end{minipage}
}

% Tirar a identação do paragrafo no texto
\def\tirarident{\setlength{\parindent}{0cm}} % padrão 15pt.

\setlength{\marginparwidth}{2cm}

%---------------------------------------------------------------
%---------------COMANDO PARA SETA DE ESCALONAMENTO--------------
%---------------------------------------------------------------

\newcommand{\seta}[3][-0.05cm]{%
  \stackon[#1]{
    $\xrightarrow{\mathmakebox[\setawidth]{}}$}{
    \scriptsize{$#2 \,\leftarrow\, #3$}
    }
}

\newcommand{\troca}[3][-0.05cm]{%
  \stackon[#1]{
    $\xrightarrow{\mathmakebox[\setawidth]{}}$}{
    \scriptsize{$#2 \,\leftrightarrow\, #3$}
    }
}

\newlength{\setawidth}% row operation width
\AtBeginDocument{\setlength{\setawidth}{2.0cm}}

%---------------------------------------------------------------
%--------------------------FIM----------------------------------
%---------------------------------------------------------------

\begin{document}
\maketitle

\vspace{-2cm}
\section*{Exercícios}

\tirarident

\textbf{Questão 03, pág. 36:}\\
\textbf{(a) $D + E$}
\begin{align*}
    D + E &=
    \underbrace{
    \begin{bmatrix}
        1 & 5 & 2\\
        -1  & 0 & 1\\
        3 & 2 & 4
    \end{bmatrix}}_{D} +
    \underbrace{
    \begin{bmatrix}
        6 & 1 & 3\\
        -1 & 1  & 2\\
        4 & 1  & 3
    \end{bmatrix}}_{E} 
    =
    \begin{bmatrix}
            1+6 & 5+2 & 2+3\\
            -1+(-1)  & 0+1 &1+2\\
            3+4 & 2+1 & 4+3
    \end{bmatrix}
    =
    \underbrace{\begin{bmatrix}
        7 & 7 & 5\\
        -2 & 1 & 3\\
        7 & 3 & 7
    \end{bmatrix}}_{D + E}
\end{align*}
    
\textbf{(b) $D - E$}
\begin{align*}
    D - E &=
    \underbrace{
    \begin{bmatrix}
        1 & 5 & 2\\
        -1  & 0 & 1\\
        3 & 2 & 4
    \end{bmatrix}}_{D} -
    \underbrace{
    \begin{bmatrix}
        6 & 1 & 3\\
        -1 & 1  & 2\\
        4 & 1  & 3
    \end{bmatrix}}_{E} 
    =
    \begin{bmatrix}
            1-6 & 5-2 & 2-3\\
            -1-(-1)  & 0-1 &1-2\\
            3-4 & 2-1 & 4-3
    \end{bmatrix}
    =
    \underbrace{\begin{bmatrix}
        -5 & 3 & -1\\
        0 & -1 & -1\\
        -1 & 1 & 1
    \end{bmatrix}}_{D - E}
\end{align*}

\textbf{(c) $5A$}
\begin{align*}
    5 \cdot A &=
    5 \cdot
    \underbrace{
    \begin{bmatrix}
        3 & 0\\
        -1 & 2\\
        1 & 1
    \end{bmatrix}}_{A} 
    =
    \begin{bmatrix}
            3 \cdot 5 & 0 \cdot 5\\
            (-1) \cdot 5 & 2 \cdot 5\\
            1 \cdot 5 & 1 \cdot 5
    \end{bmatrix}
    =
    \underbrace{\begin{bmatrix}
        15 & 0\\
        -5 & 10\\
        5 & 5
    \end{bmatrix}}_{5A}
\end{align*}

\textbf{Questão 04, pág. 36:}\\
\textbf{(a) $2A^{T} + C$}
\begin{align*}
2A^{T} + C&=
    2 \cdot
    \underbrace{
    \begin{bmatrix}
        3 & -1 & 1\\
        0 & 2 & 1
    \end{bmatrix}}_{A^{T}} 
    +
    \underbrace{
    \begin{bmatrix}
        1 & 4 & 2 \\
        3 & 1 & 5
    \end{bmatrix}}_{C} \\
    &=
    \underbrace{
    \begin{bmatrix}
        3 \cdot 2 & (-1) \cdot 2 & 1 \cdot 2\\
        0 \cdot 2 & 2 \cdot 2 & 1 \cdot 2
    \end{bmatrix}}_{2A^{T}}
    +
    \underbrace{
    \begin{bmatrix}
        1 & 4 & 2 \\
        3 & 1 & 5
    \end{bmatrix}}_{C} \\
    &=
    \underbrace{
    \begin{bmatrix}
        6 & -2 & 2\\
        0 & 4 & 2
    \end{bmatrix}}_{2A^{T}}
    +
    \underbrace{
    \begin{bmatrix}
        1 & 4 & 2 \\
        3 & 1 & 5
    \end{bmatrix}}_{C} \\
    &=
    \begin{bmatrix}
        6 + 1 & -2 + 4 & 2 + 2\\
        0 + 3 & 4 + 1 & 2 + 5
    \end{bmatrix} \\ \\
    &=
    \underbrace{
    \begin{bmatrix}
        7 & 2 & 4\\
        3 & 5 & 7
    \end{bmatrix}}_{2A^{T} + C}
\end{align*}

\textbf{(f) $B - B^{T}$}
\begin{align*}
B - B^{T}&=
    \underbrace{
    \begin{bmatrix}
        4 & -1\\
        0 & 2
    \end{bmatrix}}_{B} 
    -
    \underbrace{
    \begin{bmatrix}
        4 & 0\\
        -1 & 2
    \end{bmatrix}}_{B^{T}} \\
    &=
    \begin{bmatrix}
        4 - 4 & -1 - 0\\
        0 - (-1) & 2 - 2
    \end{bmatrix} \\ \\
    &=
    \underbrace{
    \begin{bmatrix}
        0 & -1\\
        1 & 0
    \end{bmatrix}}_{B - B^{T}} 
\end{align*}

\textbf{Questão 05, pág. 36:}\\
\textbf{(a) $AB$}
\begin{align*}
AB&=
    \underbrace{
    \begin{bmatrix}
        3 & 0\\
        -1 & 2\\
        1 & 1
    \end{bmatrix}}_{A} 
    \cdot
    \underbrace{
    \begin{bmatrix}
        4 & -1\\
        0 & 2
    \end{bmatrix}}_{B} \\
    &=
    \begin{bmatrix}
        (3\cdot 4 + 0 \cdot 0) & (3 \cdot -1 + 0 \cdot 2)\\
        ((-1) \cdot 4 + 2 \cdot 0) & ((-1) \cdot (-1) + 2 \cdot 2)\\
        (1 \cdot 4 + 1 \cdot 0) & (1 \cdot (-1) + 1 \cdot 2)
    \end{bmatrix} \\ \\
    &=
    \underbrace{
    \begin{bmatrix}
        12 & -3\\
        -4 & 5 \\
        4 & 1
    \end{bmatrix}}_{AB} 
\end{align*}

\textbf{(c) $(3E)D$}
\begin{align*}
    (3E)D&= (3 \cdot
    \underbrace{
    \begin{bmatrix}
        6 & 1 & 3\\
        -1 & 1 & 2\\
        4 & 1 & 3
    \end{bmatrix}}_{E})
    \cdot
    \underbrace{
    \begin{bmatrix}
        1 & 5 & 2\\
        -1 & 0 & 1\\
        3 & 2 & 4
    \end{bmatrix}}_{D} \\
    &=
    \begin{bmatrix}
        3 \cdot 6 & 3 \cdot 1 & 3 \cdot 3\\
        3 \cdot -1 & 3 \cdot 1 & 3 \cdot 2\\
        3 \cdot 4 & 3 \cdot 1 & 3 \cdot 3
    \end{bmatrix}
    \cdot
    \underbrace{
    \begin{bmatrix}
        1 & 5 & 2\\
        -1 & 0 & 1\\
        3 & 2 & 4
    \end{bmatrix}}_{D} \\
    &=
    \underbrace{
    \begin{bmatrix}
        18 & 3 & 9\\
        -3 & 3 & 6\\
        12 & 3 & 9
    \end{bmatrix}}_{3 \cdot E}
    \cdot
    \underbrace{
    \begin{bmatrix}
        1 & 5 & 2\\
        -1 & 0 & 1\\
        3 & 2 & 4
    \end{bmatrix}}_{D} \\
    &=
    \begin{bmatrix}
        (18\cdot 1 + 3 \cdot -1 + 9 \cdot 3) & (18 \cdot 5 + 3 \cdot 0 + 9 \cdot 2) & (18 \cdot 2 + 3 \cdot 1 + 9 \cdot 4)\\
        (-3\cdot 1 + 3 \cdot -1 + 6 \cdot 3) & (-3 \cdot 5 + 3 \cdot 0 + 6 \cdot 2) & (-3 \cdot 2 + 3 \cdot 1 + 6 \cdot 4)\\
        (12\cdot 1 + 3 \cdot -1 + 9 \cdot 3) & (12 \cdot 5 + 3 \cdot 0 + 9 \cdot 2) & (12 \cdot 2 + 3 \cdot 1 + 9 \cdot 4)
    \end{bmatrix} \\ \\
    &=
    \underbrace{
    \begin{bmatrix}
        42 & 108 & 75\\
        12 & -3 & 21\\
        36 & 78 & 63
    \end{bmatrix}}_{(3E)D} 
\end{align*}

\textbf{(d) $(AB)C$}
\begin{align*}
    (AB)C &=( 
    \underbrace{
    \begin{bmatrix}
        3 & 0\\
        -1 & 2\\
        1 & 1
    \end{bmatrix}}_{A} 
    \cdot
    \underbrace{
    \begin{bmatrix}
        4 & -1\\
        0 & 2
    \end{bmatrix}}_{B})
    \cdot
    \underbrace{
    \begin{bmatrix}
        1 & 4 & 2\\
        3 & 1 & 5
    \end{bmatrix}}_{C}
    \textrm{Aproveitando o resultado do item (A)} \\
    &=
    \underbrace{
    \begin{bmatrix}
        12 & -3\\
        -4 & 5 \\
        4 & 1
    \end{bmatrix}}_{AB}
    \cdot
    \underbrace{
    \begin{bmatrix}
        1 & 4 & 2\\
        3 & 1 & 5
    \end{bmatrix}}_{C} \\
    &=
    \begin{bmatrix}
        (12 \cdot 1 + -3 \cdot 3) & (12 \cdot 4 + -3 \cdot 1) & (12 \cdot 2 + -3 \cdot 5)\\
        (-4 \cdot 1 + 5 \cdot 3) & (-4 \cdot 4 + 5 \cdot 1) & (-4 \cdot 2 + 5 \cdot 5)\\
        (4 \cdot 1 + 1 \cdot 3) & (4 \cdot 4 + 1 \cdot 1) & (4 \cdot 2 + 1 \cdot 5)
    \end{bmatrix} \\ \\
    &=
    \underbrace{
    \begin{bmatrix}
        3 & 45 & 9 \\
        11 & -11 & 17 \\
        7 & 17 & 13
    \end{bmatrix}}_{(AB)C}
\end{align*}


\textbf{Questão 07, pág. 36:} \\
\textbf{(a) A primeira linha de AB}
\begin{align*}
    \text{1º linha de } AB &= \text{1º linha de }A \cdot B\\
    &=
    \underbrace{
    \begin{bmatrix}
        3 & -2 & 7
    \end{bmatrix}}_{\text{1º linha de }A}
    \cdot
    \underbrace{
    \begin{bmatrix}
        6 & -2 & 4 \\
        0 & 1 & 3 \\
        7 & 7 & 5
    \end{bmatrix}}_{B} \\
    &=
    \begin{bmatrix}
        3 \cdot 6    + (-2) \cdot 0 + 7 \cdot 7 &
        3 \cdot (-2) + (-2) \cdot 1 + 7 \cdot 7 &
        3 \cdot 4    + (-2) \cdot 3 + 7 \cdot 5
    \end{bmatrix}\\ \\
    &=
    \underbrace{
    \begin{bmatrix}
        67 & 41 & 41
    \end{bmatrix}}_{\text{1º linha de }AB}
\end{align*}

\textbf{(b) A terceira linha de AB}
\begin{align*}
    \text{3º linha de } AB &= \text{3º linha de }A \cdot B\\
    &=
    \underbrace{
    \begin{bmatrix}
        0 & 4 & 9
    \end{bmatrix}}_{\text{3º linha de }A}
    \cdot
    \underbrace{
    \begin{bmatrix}
        6 & -2 & 4 \\
        0 & 1 & 3 \\
        7 & 7 & 5
    \end{bmatrix}}_{B} \\
    &=
    \begin{bmatrix}
        0 \cdot 6    + 4 \cdot 0 + 9 \cdot 7 &
        0 \cdot (-2) + 4 \cdot 1 + 9 \cdot 7 &
        0 \cdot 4    + 4 \cdot 3 + 9 \cdot 5
    \end{bmatrix}\\ \\
    &=
    \underbrace{
    \begin{bmatrix}
        63 & 67 & 57
    \end{bmatrix}}_{\text{3º linha de }AB}
\end{align*}

\textbf{(c) A segunda coluna de AB}
\begin{align*}
    \text{2º coluna de } AB &= A \cdot \text{2º coluna de }B\\
    &=
    \underbrace{
    \begin{bmatrix}
        3 & -2 & 7 \\
        6 &  5 & 4 \\
        0 &  4 & 9
    \end{bmatrix}}_{A}
    \cdot
    \underbrace{
    \begin{bmatrix}
        -2 \\
         1 \\
         7
    \end{bmatrix}}_{\text{2º col. }B} \\
    &=
    \begin{bmatrix}
        3 \cdot (-2) + (-2) \cdot 1 + 7 \cdot 7 \\
        6 \cdot (-2) +   5  \cdot 1 + 4 \cdot 7 \\
        0 \cdot (-2) +   4  \cdot 1 + 9 \cdot 7
    \end{bmatrix}\\ \\
    &=
    \underbrace{
    \begin{bmatrix}
        41 \\
        21 \\
        67
    \end{bmatrix}}_{\text{2º col. }AB}
\end{align*}

\textbf{(d) A primeira coluna de BA}
\begin{align*}
    \text{1º coluna de } BA &= B \cdot \text{1º coluna de }A\\
    &=
    \underbrace{
    \begin{bmatrix}
        6 & -2 & 4 \\
        0 & 1 & 3 \\
        7 & 7 & 5
    \end{bmatrix}}_{B}
    \cdot
    \underbrace{
    \begin{bmatrix}
        3 \\
        6 \\
        0
    \end{bmatrix}}_{\text{1º col. }A} \\
    &=
    \begin{bmatrix}
        6 \cdot 3 + (-2) \cdot 6 + 4 \cdot 0 \\
        0 \cdot 3 +   1  \cdot 6 + 3 \cdot 0 \\
        7 \cdot 3 +   7  \cdot 6 + 7 \cdot 0
    \end{bmatrix} \\ \\
    &=
    \underbrace{
    \begin{bmatrix}
        6\\
        6\\
        63
    \end{bmatrix}}_{\text{1º col. }BA}
\end{align*}

\textbf{(e) A terceira linha de AA}
\begin{align*}
    \text{3º linha de }AA &= \text{3º linha de }A \cdot A \\
    &=
    \underbrace{
    \begin{bmatrix}
        0 & 4 & 9 \\
    \end{bmatrix}}_{\text{3º linha de }A}
    \cdot
    \underbrace{
    \begin{bmatrix}
        3 & -2 & 7 \\
        6 & 5 & 4 \\
        0 & 4 & 9
    \end{bmatrix}}_{A} \\
    &=
    \begin{bmatrix}
        0 \cdot 3 + 4 \cdot (-2) + 9 \cdot 7 &
        0 \cdot 6 + 4 \cdot   5  + 9 \cdot 4 &
        0 \cdot 0 + 4 \cdot   4  + 9 \cdot 9
    \end{bmatrix} \\ \\
    &=
    \underbrace{
    \begin{bmatrix}
        55 & 56 & 97
    \end{bmatrix}}_{\text{3º linha de }AA}
\end{align*}

\textbf{(e) A terceira coluna de AA}
\begin{align*}
    \text{3º coluna de }AA &= A \cdot \text{3º coluna de }A \\
    &=
    \underbrace{
    \begin{bmatrix}
        3 & -2 & 7 \\
        6 & 5 & 4 \\
        0 & 4 & 9
    \end{bmatrix}}_{A}
    \cdot
    \underbrace{
    \begin{bmatrix}
        7 \\
        4 \\
        9
    \end{bmatrix}}_{\text{3º coluna de }A} \\
    &=
    \begin{bmatrix}
        3 \cdot 7 + (-2) \cdot 4 + 7 \cdot 9 \\
        6 \cdot 7 +   5  \cdot 4 + 4 \cdot 9 \\
        0 \cdot 7 +   4  \cdot 4 + 9 \cdot 9
    \end{bmatrix} \\ \\
    &=
    \underbrace{
    \begin{bmatrix}
        76 & 98 & 97
    \end{bmatrix}}_{\text{3º coluna de }AA}
\end{align*}

\textbf{Questão 10, pág. 36:}\\
\textbf{(a) Expresse cada vetor coluna de $AB$ como uma combinação linear dos vetores coluna de $A$}
\begin{align*}
    \text{1º coluna de AB} =
    6 &\cdot
    \begin{bmatrix}
        3 \\
        6 \\
        0
    \end{bmatrix}
    + 0 \cdot 
    \begin{bmatrix}
        -2 \\
         5 \\
         4
    \end{bmatrix} 
    + 7 \cdot
    \begin{bmatrix}
        7 \\
        4 \\
        9
    \end{bmatrix} \\
    \text{2º coluna de AB} =
    (-2) &\cdot
    \begin{bmatrix}
        3 \\
        6 \\
        0
    \end{bmatrix}
    + 1 \cdot 
    \begin{bmatrix}
        -2 \\
         5 \\
         4
    \end{bmatrix} 
    + 7 \cdot
    \begin{bmatrix}
        7 \\
        4 \\
        9
    \end{bmatrix} \\
    \text{3º coluna de AB} =
    7 &\cdot
    \begin{bmatrix}
        3 \\
        6 \\
        0
    \end{bmatrix}
    + 4 \cdot 
    \begin{bmatrix}
        -2 \\
         5 \\
         4
    \end{bmatrix} 
    + 9 \cdot
    \begin{bmatrix}
        7 \\
        4 \\
        9
    \end{bmatrix}
\end{align*}

\textbf{(b) Expresse cada vetor coluna de BA como uma combinação linear dos vetores coluna de B}
\begin{align*}
    \text{1º coluna de BA} =
    3 \cdot
    \begin{bmatrix}
        6 \\
        0 \\
        7
    \end{bmatrix}
    + (-2) &\cdot 
    \begin{bmatrix}
        -2 \\
         1 \\
         7
    \end{bmatrix} 
    + 7 \cdot
    \begin{bmatrix}
        4 \\
        3 \\
        5
    \end{bmatrix} \\ 
    \text{2º coluna de BA} =
    6 \cdot
    \begin{bmatrix}
        6 \\
        0 \\
        7
    \end{bmatrix}
    + 5 &\cdot 
    \begin{bmatrix}
        -2 \\
         1 \\
         7
    \end{bmatrix} 
    + 4 \cdot
    \begin{bmatrix}
        4 \\
        3 \\
        5
    \end{bmatrix} \\ 
    \text{3º coluna de BA} =
    0 \cdot
    \begin{bmatrix}
        6 \\
        0 \\
        7
    \end{bmatrix}
    + 4 &\cdot 
    \begin{bmatrix}
        -2 \\
         1 \\
         7
    \end{bmatrix} 
    + 9 \cdot
    \begin{bmatrix}
        4 \\
        3 \\
        5
    \end{bmatrix} \\ 
\end{align*}

\textbf{Questão 11, pág. 36:}\\
\textbf{(a)}
\begin{align*}
    \underbrace{
    \begin{bmatrix}
        2 & -3 & 5\\
        9 & -1 & 1\\
        1 & 5 & 4
    \end{bmatrix}}_{A} 
    \cdot
    \underbrace{
    \begin{bmatrix}
        x_{1} \\
        x_{2} \\
        x_{3}
    \end{bmatrix}}_{x}
    =
    \underbrace{
    \begin{bmatrix}
        7 \\
        -1 \\
        0
    \end{bmatrix}}_{B}
\end{align*}

\textbf{(b)}
\begin{align*}
    \underbrace{
    \begin{bmatrix}
        4 & 0 & -3 & 1\\
        5 & 1 & 0 & -8\\
        2 & -5 & 9 & 4\\
        0 & 3 & -1 & 7
    \end{bmatrix}}_{A} 
    \cdot
    \underbrace{
    \begin{bmatrix}
        x_{1} \\
        x_{2} \\
        x_{3} \\
        x_{4}
    \end{bmatrix}}_{x}
    =
    \underbrace{
    \begin{bmatrix}
        1 \\
        3 \\
        0 \\
        2
    \end{bmatrix}}_{B}
\end{align*}

\textbf{Questão 13, pág. 36:}\\
\textbf{(a)}
\begin{equation*}
    \begin{cases*}
        5x_1 + 6x_2 - 7x_3 = 2 \\
        -x_1 - 2x_2 + 3x_3 = 0 \\
               4x_2 - x_ 3 = 3
    \end{cases*}
\end{equation*}
\textbf{(b)}
\begin{equation*}
    \begin{cases*}
         x_1 +  x_2 +  x_3 = 2 \\
        2x_1 + 3x_2        = 2 \\
        5x_1 - 3x_2 - 6x_3 = 3
    \end{cases*}
\end{equation*}

\end{document}